% ****************************************************************************************************
% classicthesis-config.tex
% formerly known as loadpackages.sty, classicthesis-ldpkg.sty, and classicthesis-preamble.sty
% Use it at the beginning of your ClassicThesis.tex, or as a LaTeX Preamble
% in your ClassicThesis.{tex,lyx} with % ****************************************************************************************************
% classicthesis-config.tex
% formerly known as loadpackages.sty, classicthesis-ldpkg.sty, and classicthesis-preamble.sty
% Use it at the beginning of your ClassicThesis.tex, or as a LaTeX Preamble
% in your ClassicThesis.{tex,lyx} with % ****************************************************************************************************
% classicthesis-config.tex
% formerly known as loadpackages.sty, classicthesis-ldpkg.sty, and classicthesis-preamble.sty
% Use it at the beginning of your ClassicThesis.tex, or as a LaTeX Preamble
% in your ClassicThesis.{tex,lyx} with % ****************************************************************************************************
% classicthesis-config.tex
% formerly known as loadpackages.sty, classicthesis-ldpkg.sty, and classicthesis-preamble.sty
% Use it at the beginning of your ClassicThesis.tex, or as a LaTeX Preamble
% in your ClassicThesis.{tex,lyx} with \input{classicthesis-config}
% ****************************************************************************************************
% If you like the classicthesis, then I would appreciate a postcard.
% My address can be found in the file ClassicThesis.pdf. A collection
% of the postcards I received so far is available online at
% http://postcards.miede.de
% ****************************************************************************************************


% ****************************************************************************************************
% 0. Set the encoding of your files. UTF-8 is the only sensible encoding nowadays. If you can't read
% äöüßáéçèê∂åëæƒÏ€ then change the encoding setting in your editor, not the line below. If your editor
% does not support utf8 use another editor!
% ****************************************************************************************************
\PassOptionsToPackage{utf8}{inputenc}
\usepackage{inputenc}

\PassOptionsToPackage{T1}{fontenc} % T2A for cyrillics
\usepackage{fontenc}


% ****************************************************************************************************
% 1. Configure classicthesis for your needs here, e.g., remove "drafting" below
% in order to deactivate the time-stamp on the pages
% (see ClassicThesis.pdf for more information):
% ****************************************************************************************************
\PassOptionsToPackage{
	drafting=false,    % print version information on the bottom of the pages
	tocaligned=false, % the left column of the toc will be aligned (no indentation)
	dottedtoc=false,  % page numbers in ToC flushed right
	eulerchapternumbers=true, % use AMS Euler for chapter font (otherwise Palatino)
	linedheaders=false,       % chaper headers will have line above and beneath
	floatperchapter=true,     % numbering per chapter for all floats (i.e., Figure 1.1)
	eulermath=false,  % use awesome Euler fonts for mathematical formulae (only with pdfLaTeX)
	beramono=false,    % toggle a nice monospaced font (w/ bold)
	palatino=false,    % deactivate standard font for loading another one, see the last section at the end of this file for suggestions
	style=classicthesis % classicthesis, arsclassica
}{classicthesis}


% ****************************************************************************************************
% 2. Personal data and user ad-hoc commands (insert your own data here)
% ****************************************************************************************************
\newcommand{\myTitle}{Advances in Recommender Systems for Some Applications \xspace}
\newcommand{\mySubtitle}{A thesis about some recommender systems stuff\xspace}
\newcommand{\myDegree}{PhD\xspace}
\newcommand{\myStudentNumber}{18201111\xspace}
\newcommand{\myName}{Eolas MacDalta\xspace}
\newcommand{\myProf}{Put name here\xspace}
\newcommand{\myOtherProf}{Put name here\xspace}
\newcommand{\mySupervisor}{Prof. A.N Supervisor\xspace}
\newcommand{\myCoSupervisor}{Prof. A.N CoSupervisor\xspace}
\newcommand{\myFaculty}{Put data here\xspace}
\newcommand{\myHOS}{Prof. H. O'Scoil\xspace}
\newcommand{\myRSPA}{Prof. A. B. RSPHead\xspace}
\newcommand{\myRSPB}{Prof. B. C. RSPB\xspace}
\newcommand{\myDepartment}{School of Computer Science\xspace}
%\newcommand{\myTitleDeclaration}{A thesis submitted to University College Dublin in fulfilment of the requirements for the degree of Doctor of Philosophy\xspace}
\newcommand{\myTitleDeclaration}{This thesis is submitted to University College Dublin in fulfilment of the requirements for the degree of Doctor of Philosophy\xspace}
%\newcommand{\myFaculty}{Put data here\xspace}
\newcommand{\myUni}{University College Dublin\xspace}
\newcommand{\myLocation}{University College Dublin\xspace}
\newcommand{\myTime}{January, 2024\xspace}
\newcommand{\myVersion}{version 1.0\xspace}


% ********************************************************************
% Setup, finetuning, and useful commands
% ********************************************************************
\providecommand{\mLyX}{L\kern-.1667em\lower.25em\hbox{Y}\kern-.125emX\@}
\newcommand{\ie}{i.\,e.}
\newcommand{\Ie}{I.\,e.}
\newcommand{\eg}{e.\,g.}
\newcommand{\Eg}{E.\,g.}
\newcommand\cleartooddpage{\clearpage
	\ifodd\value{page}\else\null\clearpage\fi}
% ****************************************************************************************************


% ****************************************************************************************************
% 3. Loading some handy packages
% ****************************************************************************************************
% ********************************************************************
% Packages with options that might require adjustments
% ********************************************************************
\PassOptionsToPackage{english}{babel} % change this to your language(s), main language last
% Spanish languages need extra options in order to work with this template
%\PassOptionsToPackage{spanish,es-lcroman}{babel}
\usepackage{babel}

\usepackage{csquotes}
\PassOptionsToPackage{%
	%backend=biber,bibencoding=utf8, %instead of bibtex
	backend=bibtex8,bibencoding=ascii,%
	language=auto,%
	style=numeric-comp,%
	%style=authoryear-comp, % Author 1999, 2010
	%bibstyle=authoryear,dashed=false, % dashed: substitute rep. author with ---
	sorting=nyt, % name, year, title
	maxbibnames=10, % default: 3, et al.
	%backref=true,%
	natbib=true % natbib compatibility mode (\citep and \citet still work)
}{biblatex}
\usepackage{biblatex}

\PassOptionsToPackage{fleqn}{amsmath}       % math environments and more by the AMS
\usepackage{amsmath}

% ********************************************************************
% General useful packages
% ********************************************************************
\usepackage{graphicx} %
\usepackage[export]{adjustbox}
\usepackage{scrhack} % fix warnings when using KOMA with listings package
\usepackage{xspace} % to get the spacing after macros right
\PassOptionsToPackage{printonlyused,smaller}{acronym}
\usepackage{acronym} % nice macros for handling all acronyms in the thesis
%\renewcommand{\bflabel}[1]{{#1}\hfill} % fix the list of acronyms --> no longer working
%\renewcommand*{\acsfont}[1]{\textsc{#1}}
%\renewcommand*{\aclabelfont}[1]{\acsfont{#1}}
%\def\bflabel#1{{#1\hfill}}
\def\bflabel#1{{\acsfont{#1}\hfill}}
\def\aclabelfont#1{\acsfont{#1}}
% ****************************************************************************************************
%\usepackage{pgfplots} % External TikZ/PGF support (thanks to Andreas Nautsch)
%\usetikzlibrary{external}
%\tikzexternalize[mode=list and make, prefix=ext-tikz/]
% ****************************************************************************************************


% ****************************************************************************************************
% 4. Setup floats: tables, (sub)figures, and captions
% ****************************************************************************************************
\usepackage{tabularx} % better tables
\setlength{\extrarowheight}{3pt} % increase table row height
\newcommand{\tableheadline}[1]{\multicolumn{1}{l}{\spacedlowsmallcaps{#1}}}
\newcommand{\myfloatalign}{\centering} % to be used with each float for alignment
\usepackage{subfig}


\usepackage{blindtext}
% ****************************************************************************************************


% ****************************************************************************************************
% 5. Setup code listings
% ****************************************************************************************************
\usepackage{listings}
%\lstset{emph={trueIndex,root},emphstyle=\color{BlueViolet}}%\underbar} % for special keywords
\lstset{language=[LaTeX]Tex,%C++,
	morekeywords={PassOptionsToPackage,selectlanguage},
	keywordstyle=\color{RoyalBlue},%\bfseries,
	basicstyle=\small\ttfamily,
	%identifierstyle=\color{NavyBlue},
	commentstyle=\color{Green}\ttfamily,
	stringstyle=\rmfamily,
	numbers=none,%left,%
	numberstyle=\scriptsize,%\tiny
	stepnumber=5,
	numbersep=8pt,
	showstringspaces=false,
	breaklines=true,
	%frameround=ftff,
	%frame=single,
	belowcaptionskip=.75\baselineskip
	%frame=L
}

\definecolor{commentcolor}{rgb}{0.2,0.4,0.2}
\lstset{
    float,
    % By default, {listings} uses proportional-width fonts, but Source
    % Code Pro is nice enough to use on its own. Thus, we override basicstyle
    % as using the monospace font we declared earlier with {fontspec}.
    basicstyle=\ttfamily,
    % Also by default, everything is just black. Let's add some color to
    % comments.
    commentstyle=\small\color{commentcolor}\ttfamily,
    % The following few settings set the listings apart using a line
    % at the top and bottom, and subtle line numbers to the left.
    numbers=right,
    numberstyle=\tiny, stepnumber=5,
    numbersep=5pt,
    frame=lines,
    % Here's where the real magic is: mathescape. We want to catch both
    % Sphinx-style and LaTeX-style math escapes, which is a bit of a hack.
    % Just don't try and use this to typeset JavaScript code that touches
    % jQuery...
    mathescape=true,
    escapeinside={:math:`}{`},
    escapebegin=$,
    escapeend=$
}

% We'll go on and set a few options specific to Python, too, and encapsulate
% them into a {listings} style. This way, if we have a multilingual document,
% it will play nice.
\lstdefinestyle{oldpython}{
    language=Python,
    tabsize=4,
    showstringspaces=false
}

\definecolor{deepgreen}{rgb}{0,0.4,0}

\lstdefinestyle{python}{
		language=Python,
		basicstyle=\color{black}\ttfamily\footnotesize,
		stringstyle=\color{deepgreen}\slshape,
		commentstyle=\color{gray}\slshape,
		keywordstyle=\color{red}\textbf,
		emphstyle=\color{blue}\textbf,
		tabsize=2,
		%%%%%%%%%%%%%%%
		showstringspaces=false,
    emph={access,and,break,class,continue,def,del,elif,else,except,exec,finally,for,from,global,if,import,in,i s,lambda,not,or,pass,print,raise,return,try,while,as},
		upquote=true,
		morecomment=[s]{"""}{"""},
		literate=*
		{:}{{\textcolor{blue}:}}{1}%
		{=}{{\textcolor{blue}=}}{1}%
		{-}{{\textcolor{blue}-}}{1}%
		{+}{{\textcolor{blue}+}}{1}%
		{*}{{\textcolor{blue}*}}{1}%
		{!}{{\textcolor{blue}!}}{1}%
		{(}{{\textcolor{blue}(}}{1}%
		{)}{{\textcolor{blue})}}{1}%
		{[}{{\textcolor{blue}[}}{1}%
		{]}{{\textcolor{blue}]}}{1}%
		{<}{{\textcolor{blue}<}}{1}%
		{>}{{\textcolor{blue}>}}{1},%
		%%%%%%%%%%%%%%%%
		aboveskip=\baselineskip,
		xleftmargin=20pt, xrightmargin=15pt,
		frame=lines,
		numbers=left, numberstyle=\tiny
}

% ****************************************************************************************************




% ****************************************************************************************************
% 6. Last calls before the bar closes
% ****************************************************************************************************
\usepackage{afterpage}
% ********************************************************************
% Her Majesty herself
% ********************************************************************
\usepackage{classicthesis}

\clearscrheadfoot
\ohead[]{\headmark}
\rofoot[\pagemark]{\pagemark}
\lefoot[\pagemark]{\pagemark}

% ********************************************************************
% Fine-tune hyperreferences (hyperref should be called last)
% ********************************************************************
\hypersetup{%
	%draft, % hyperref's draft mode, for printing see below
	colorlinks=true, linktocpage=true, pdfstartpage=3, pdfstartview=FitV,%
	% uncomment the following line if you want to have black links (e.g., for printing)
	%colorlinks=false, linktocpage=false, pdfstartpage=3, pdfstartview=FitV, pdfborder={0 0 0},%
	breaklinks=true, pageanchor=true,%
	pdfpagemode=UseNone, %
	% pdfpagemode=UseOutlines,%
	plainpages=false, bookmarksnumbered, bookmarksopen=true, bookmarksopenlevel=1,%
	hypertexnames=true, pdfhighlight=/O,%nesting=true,%frenchlinks,%
	urlcolor=CTurl, linkcolor=CTlink, citecolor=CTcitation, %pagecolor=RoyalBlue,%
	%urlcolor=Black, linkcolor=Black, citecolor=Black, %pagecolor=Black,%
	pdftitle={\myTitle},%
	pdfauthor={\textcopyright\ \myName, \myUni, \myFaculty},%
	pdfsubject={},%
	pdfkeywords={},%
	pdfcreator={pdfLaTeX},%
	pdfproducer={LaTeX with hyperref and classicthesis}%
}


% ********************************************************************
% Setup autoreferences (hyperref and babel)
% ********************************************************************
% There are some issues regarding autorefnames
% http://www.tex.ac.uk/cgi-bin/texfaq2html?label=latexwords
% you have to redefine the macros for the
% language you use, e.g., american, ngerman
% (as chosen when loading babel/AtBeginDocument)
% ********************************************************************
\makeatletter
\@ifpackageloaded{babel}%
{%
	\addto\extrasenglish{%
		\renewcommand*{\figureautorefname}{Figure}%
		\renewcommand*{\tableautorefname}{Table}%
		\renewcommand*{\partautorefname}{Part}%
		\renewcommand*{\chapterautorefname}{Chapter}%
		\renewcommand*{\sectionautorefname}{Section}%
		\renewcommand*{\subsectionautorefname}{Section}%
		\renewcommand*{\subsubsectionautorefname}{Section}%
	}%
	\addto\extrasngerman{%
		\renewcommand*{\paragraphautorefname}{Absatz}%
		\renewcommand*{\subparagraphautorefname}{Unterabsatz}%
		\renewcommand*{\footnoteautorefname}{Fu\"snote}%
		\renewcommand*{\FancyVerbLineautorefname}{Zeile}%
		\renewcommand*{\theoremautorefname}{Theorem}%
		\renewcommand*{\appendixautorefname}{Anhang}%
		\renewcommand*{\equationautorefname}{Gleichung}%
		\renewcommand*{\itemautorefname}{Punkt}%
	}%
	% Fix to getting autorefs for subfigures right (thanks to Belinda Vogt for changing the definition)
	\providecommand{\subfigureautorefname}{\figureautorefname}%
}{\relax}
\makeatother


%
%
%
%\PassOptionsToPackage{}{glossaries}
\usepackage{glossaries}

% ********************************************************************
% Development Stuff
% ********************************************************************
\listfiles


% ****************************************************************************************************
% 7. Further adjustments (experimental)
% ****************************************************************************************************
% ********************************************************************
% Changing the text area
% ********************************************************************
%\areaset[current]{312pt}{761pt} % 686 (factor 2.2) + 33 head + 42 head \the\footskip
%\setlength{\marginparwidth}{7em}%
%\setlength{\marginparsep}{2em}%

%****
% colors
%***
\definecolor{UCDBlue}{RGB}{0, 108, 174}
\definecolor{UCDDarkBlue}{RGB}{16, 55, 96}

% ********************************************************************
% Using different fonts
% ********************************************************************
%\PassOptionsToPackage{doublespacing}{setspace}
\PassOptionsToPackage{onehalfspacing}{setspace}
\usepackage{setspace}
\setstretch{1.5}

\PassOptionsToPackage{switch}{lineno}
\usepackage{lineno}
\renewcommand\linenumberfont{\normalfont\fontsize{6}{0}\selectfont\color{Gray}}


\ifthenelse{\boolean{xetex}\OR\boolean{luatex}}
{
	% have to use sans serif fonts
	\RequirePackage{fontspec}
	\usepackage[varqu,varl]{zi4}% inconsolata typewriter
	\usepackage{amsmath,amsthm}
	\usepackage{libertinus}

	\newfontfamily\universityfont{Cabin}
	\newfontfamily\textuniversityfont{Cabin}
	\newfontfamily\texttitlepagefont{EB Garamond}
	\newfontfamily\textthesistitlefont{Latin Modern Sans}

	\setmainfont[Mapping=tex-text]{Libertinus Sans}
  \setmonofont{Fira Mono}
	\renewcommand{\familydefault}{\sfdefault}
}
{ % use Type 1 fonts with pdflatex
	\usepackage{librecaslon}
	\usepackage[T1]{fontenc}
	\usepackage{ebgaramond-maths}
	\usepackage{cabin}
	\usepackage{libertinus}
	\usepackage[varqu]{zi4}
	\usepackage{libertinust1math}

	\renewcommand*\familydefault{\sfdefault}  %% Only if the base font of the document is to be sans serif

	\DeclareTextFontCommand{\textuniversityfont}{
		\fontfamily{Cabin-TLF}\selectfont
	}

	\DeclareTextFontCommand{\texttitlepagefont}{
		\fontfamily{EBGaramond-TLF}\selectfont
	}
}


\usepackage[left=3.2cm, right=3.2cm, top=2cm, bottom=2.5cm, includehead, includefoot]{geometry}

\usepackage{algorithm}
\usepackage{algorithmicx}
\usepackage{algpseudocode}
% ****************************************************************************************************
% If you like the classicthesis, then I would appreciate a postcard.
% My address can be found in the file ClassicThesis.pdf. A collection
% of the postcards I received so far is available online at
% http://postcards.miede.de
% ****************************************************************************************************


% ****************************************************************************************************
% 0. Set the encoding of your files. UTF-8 is the only sensible encoding nowadays. If you can't read
% äöüßáéçèê∂åëæƒÏ€ then change the encoding setting in your editor, not the line below. If your editor
% does not support utf8 use another editor!
% ****************************************************************************************************
\PassOptionsToPackage{utf8}{inputenc}
\usepackage{inputenc}

\PassOptionsToPackage{T1}{fontenc} % T2A for cyrillics
\usepackage{fontenc}


% ****************************************************************************************************
% 1. Configure classicthesis for your needs here, e.g., remove "drafting" below
% in order to deactivate the time-stamp on the pages
% (see ClassicThesis.pdf for more information):
% ****************************************************************************************************
\PassOptionsToPackage{
	drafting=false,    % print version information on the bottom of the pages
	tocaligned=false, % the left column of the toc will be aligned (no indentation)
	dottedtoc=false,  % page numbers in ToC flushed right
	eulerchapternumbers=true, % use AMS Euler for chapter font (otherwise Palatino)
	linedheaders=false,       % chaper headers will have line above and beneath
	floatperchapter=true,     % numbering per chapter for all floats (i.e., Figure 1.1)
	eulermath=false,  % use awesome Euler fonts for mathematical formulae (only with pdfLaTeX)
	beramono=false,    % toggle a nice monospaced font (w/ bold)
	palatino=false,    % deactivate standard font for loading another one, see the last section at the end of this file for suggestions
	style=classicthesis % classicthesis, arsclassica
}{classicthesis}


% ****************************************************************************************************
% 2. Personal data and user ad-hoc commands (insert your own data here)
% ****************************************************************************************************
\newcommand{\myTitle}{Advances in Recommender Systems for Some Applications \xspace}
\newcommand{\mySubtitle}{A thesis about some recommender systems stuff\xspace}
\newcommand{\myDegree}{PhD\xspace}
\newcommand{\myStudentNumber}{18201111\xspace}
\newcommand{\myName}{Eolas MacDalta\xspace}
\newcommand{\myProf}{Put name here\xspace}
\newcommand{\myOtherProf}{Put name here\xspace}
\newcommand{\mySupervisor}{Prof. A.N Supervisor\xspace}
\newcommand{\myCoSupervisor}{Prof. A.N CoSupervisor\xspace}
\newcommand{\myFaculty}{Put data here\xspace}
\newcommand{\myHOS}{Prof. H. O'Scoil\xspace}
\newcommand{\myRSPA}{Prof. A. B. RSPHead\xspace}
\newcommand{\myRSPB}{Prof. B. C. RSPB\xspace}
\newcommand{\myDepartment}{School of Computer Science\xspace}
%\newcommand{\myTitleDeclaration}{A thesis submitted to University College Dublin in fulfilment of the requirements for the degree of Doctor of Philosophy\xspace}
\newcommand{\myTitleDeclaration}{This thesis is submitted to University College Dublin in fulfilment of the requirements for the degree of Doctor of Philosophy\xspace}
%\newcommand{\myFaculty}{Put data here\xspace}
\newcommand{\myUni}{University College Dublin\xspace}
\newcommand{\myLocation}{University College Dublin\xspace}
\newcommand{\myTime}{January, 2024\xspace}
\newcommand{\myVersion}{version 1.0\xspace}


% ********************************************************************
% Setup, finetuning, and useful commands
% ********************************************************************
\providecommand{\mLyX}{L\kern-.1667em\lower.25em\hbox{Y}\kern-.125emX\@}
\newcommand{\ie}{i.\,e.}
\newcommand{\Ie}{I.\,e.}
\newcommand{\eg}{e.\,g.}
\newcommand{\Eg}{E.\,g.}
\newcommand\cleartooddpage{\clearpage
	\ifodd\value{page}\else\null\clearpage\fi}
% ****************************************************************************************************


% ****************************************************************************************************
% 3. Loading some handy packages
% ****************************************************************************************************
% ********************************************************************
% Packages with options that might require adjustments
% ********************************************************************
\PassOptionsToPackage{english}{babel} % change this to your language(s), main language last
% Spanish languages need extra options in order to work with this template
%\PassOptionsToPackage{spanish,es-lcroman}{babel}
\usepackage{babel}

\usepackage{csquotes}
\PassOptionsToPackage{%
	%backend=biber,bibencoding=utf8, %instead of bibtex
	backend=bibtex8,bibencoding=ascii,%
	language=auto,%
	style=numeric-comp,%
	%style=authoryear-comp, % Author 1999, 2010
	%bibstyle=authoryear,dashed=false, % dashed: substitute rep. author with ---
	sorting=nyt, % name, year, title
	maxbibnames=10, % default: 3, et al.
	%backref=true,%
	natbib=true % natbib compatibility mode (\citep and \citet still work)
}{biblatex}
\usepackage{biblatex}

\PassOptionsToPackage{fleqn}{amsmath}       % math environments and more by the AMS
\usepackage{amsmath}

% ********************************************************************
% General useful packages
% ********************************************************************
\usepackage{graphicx} %
\usepackage[export]{adjustbox}
\usepackage{scrhack} % fix warnings when using KOMA with listings package
\usepackage{xspace} % to get the spacing after macros right
\PassOptionsToPackage{printonlyused,smaller}{acronym}
\usepackage{acronym} % nice macros for handling all acronyms in the thesis
%\renewcommand{\bflabel}[1]{{#1}\hfill} % fix the list of acronyms --> no longer working
%\renewcommand*{\acsfont}[1]{\textsc{#1}}
%\renewcommand*{\aclabelfont}[1]{\acsfont{#1}}
%\def\bflabel#1{{#1\hfill}}
\def\bflabel#1{{\acsfont{#1}\hfill}}
\def\aclabelfont#1{\acsfont{#1}}
% ****************************************************************************************************
%\usepackage{pgfplots} % External TikZ/PGF support (thanks to Andreas Nautsch)
%\usetikzlibrary{external}
%\tikzexternalize[mode=list and make, prefix=ext-tikz/]
% ****************************************************************************************************


% ****************************************************************************************************
% 4. Setup floats: tables, (sub)figures, and captions
% ****************************************************************************************************
\usepackage{tabularx} % better tables
\setlength{\extrarowheight}{3pt} % increase table row height
\newcommand{\tableheadline}[1]{\multicolumn{1}{l}{\spacedlowsmallcaps{#1}}}
\newcommand{\myfloatalign}{\centering} % to be used with each float for alignment
\usepackage{subfig}


\usepackage{blindtext}
% ****************************************************************************************************


% ****************************************************************************************************
% 5. Setup code listings
% ****************************************************************************************************
\usepackage{listings}
%\lstset{emph={trueIndex,root},emphstyle=\color{BlueViolet}}%\underbar} % for special keywords
\lstset{language=[LaTeX]Tex,%C++,
	morekeywords={PassOptionsToPackage,selectlanguage},
	keywordstyle=\color{RoyalBlue},%\bfseries,
	basicstyle=\small\ttfamily,
	%identifierstyle=\color{NavyBlue},
	commentstyle=\color{Green}\ttfamily,
	stringstyle=\rmfamily,
	numbers=none,%left,%
	numberstyle=\scriptsize,%\tiny
	stepnumber=5,
	numbersep=8pt,
	showstringspaces=false,
	breaklines=true,
	%frameround=ftff,
	%frame=single,
	belowcaptionskip=.75\baselineskip
	%frame=L
}

\definecolor{commentcolor}{rgb}{0.2,0.4,0.2}
\lstset{
    float,
    % By default, {listings} uses proportional-width fonts, but Source
    % Code Pro is nice enough to use on its own. Thus, we override basicstyle
    % as using the monospace font we declared earlier with {fontspec}.
    basicstyle=\ttfamily,
    % Also by default, everything is just black. Let's add some color to
    % comments.
    commentstyle=\small\color{commentcolor}\ttfamily,
    % The following few settings set the listings apart using a line
    % at the top and bottom, and subtle line numbers to the left.
    numbers=right,
    numberstyle=\tiny, stepnumber=5,
    numbersep=5pt,
    frame=lines,
    % Here's where the real magic is: mathescape. We want to catch both
    % Sphinx-style and LaTeX-style math escapes, which is a bit of a hack.
    % Just don't try and use this to typeset JavaScript code that touches
    % jQuery...
    mathescape=true,
    escapeinside={:math:`}{`},
    escapebegin=$,
    escapeend=$
}

% We'll go on and set a few options specific to Python, too, and encapsulate
% them into a {listings} style. This way, if we have a multilingual document,
% it will play nice.
\lstdefinestyle{oldpython}{
    language=Python,
    tabsize=4,
    showstringspaces=false
}

\definecolor{deepgreen}{rgb}{0,0.4,0}

\lstdefinestyle{python}{
		language=Python,
		basicstyle=\color{black}\ttfamily\footnotesize,
		stringstyle=\color{deepgreen}\slshape,
		commentstyle=\color{gray}\slshape,
		keywordstyle=\color{red}\textbf,
		emphstyle=\color{blue}\textbf,
		tabsize=2,
		%%%%%%%%%%%%%%%
		showstringspaces=false,
    emph={access,and,break,class,continue,def,del,elif,else,except,exec,finally,for,from,global,if,import,in,i s,lambda,not,or,pass,print,raise,return,try,while,as},
		upquote=true,
		morecomment=[s]{"""}{"""},
		literate=*
		{:}{{\textcolor{blue}:}}{1}%
		{=}{{\textcolor{blue}=}}{1}%
		{-}{{\textcolor{blue}-}}{1}%
		{+}{{\textcolor{blue}+}}{1}%
		{*}{{\textcolor{blue}*}}{1}%
		{!}{{\textcolor{blue}!}}{1}%
		{(}{{\textcolor{blue}(}}{1}%
		{)}{{\textcolor{blue})}}{1}%
		{[}{{\textcolor{blue}[}}{1}%
		{]}{{\textcolor{blue}]}}{1}%
		{<}{{\textcolor{blue}<}}{1}%
		{>}{{\textcolor{blue}>}}{1},%
		%%%%%%%%%%%%%%%%
		aboveskip=\baselineskip,
		xleftmargin=20pt, xrightmargin=15pt,
		frame=lines,
		numbers=left, numberstyle=\tiny
}

% ****************************************************************************************************




% ****************************************************************************************************
% 6. Last calls before the bar closes
% ****************************************************************************************************
\usepackage{afterpage}
% ********************************************************************
% Her Majesty herself
% ********************************************************************
\usepackage{classicthesis}

\clearscrheadfoot
\ohead[]{\headmark}
\rofoot[\pagemark]{\pagemark}
\lefoot[\pagemark]{\pagemark}

% ********************************************************************
% Fine-tune hyperreferences (hyperref should be called last)
% ********************************************************************
\hypersetup{%
	%draft, % hyperref's draft mode, for printing see below
	colorlinks=true, linktocpage=true, pdfstartpage=3, pdfstartview=FitV,%
	% uncomment the following line if you want to have black links (e.g., for printing)
	%colorlinks=false, linktocpage=false, pdfstartpage=3, pdfstartview=FitV, pdfborder={0 0 0},%
	breaklinks=true, pageanchor=true,%
	pdfpagemode=UseNone, %
	% pdfpagemode=UseOutlines,%
	plainpages=false, bookmarksnumbered, bookmarksopen=true, bookmarksopenlevel=1,%
	hypertexnames=true, pdfhighlight=/O,%nesting=true,%frenchlinks,%
	urlcolor=CTurl, linkcolor=CTlink, citecolor=CTcitation, %pagecolor=RoyalBlue,%
	%urlcolor=Black, linkcolor=Black, citecolor=Black, %pagecolor=Black,%
	pdftitle={\myTitle},%
	pdfauthor={\textcopyright\ \myName, \myUni, \myFaculty},%
	pdfsubject={},%
	pdfkeywords={},%
	pdfcreator={pdfLaTeX},%
	pdfproducer={LaTeX with hyperref and classicthesis}%
}


% ********************************************************************
% Setup autoreferences (hyperref and babel)
% ********************************************************************
% There are some issues regarding autorefnames
% http://www.tex.ac.uk/cgi-bin/texfaq2html?label=latexwords
% you have to redefine the macros for the
% language you use, e.g., american, ngerman
% (as chosen when loading babel/AtBeginDocument)
% ********************************************************************
\makeatletter
\@ifpackageloaded{babel}%
{%
	\addto\extrasenglish{%
		\renewcommand*{\figureautorefname}{Figure}%
		\renewcommand*{\tableautorefname}{Table}%
		\renewcommand*{\partautorefname}{Part}%
		\renewcommand*{\chapterautorefname}{Chapter}%
		\renewcommand*{\sectionautorefname}{Section}%
		\renewcommand*{\subsectionautorefname}{Section}%
		\renewcommand*{\subsubsectionautorefname}{Section}%
	}%
	\addto\extrasngerman{%
		\renewcommand*{\paragraphautorefname}{Absatz}%
		\renewcommand*{\subparagraphautorefname}{Unterabsatz}%
		\renewcommand*{\footnoteautorefname}{Fu\"snote}%
		\renewcommand*{\FancyVerbLineautorefname}{Zeile}%
		\renewcommand*{\theoremautorefname}{Theorem}%
		\renewcommand*{\appendixautorefname}{Anhang}%
		\renewcommand*{\equationautorefname}{Gleichung}%
		\renewcommand*{\itemautorefname}{Punkt}%
	}%
	% Fix to getting autorefs for subfigures right (thanks to Belinda Vogt for changing the definition)
	\providecommand{\subfigureautorefname}{\figureautorefname}%
}{\relax}
\makeatother


%
%
%
%\PassOptionsToPackage{}{glossaries}
\usepackage{glossaries}

% ********************************************************************
% Development Stuff
% ********************************************************************
\listfiles


% ****************************************************************************************************
% 7. Further adjustments (experimental)
% ****************************************************************************************************
% ********************************************************************
% Changing the text area
% ********************************************************************
%\areaset[current]{312pt}{761pt} % 686 (factor 2.2) + 33 head + 42 head \the\footskip
%\setlength{\marginparwidth}{7em}%
%\setlength{\marginparsep}{2em}%

%****
% colors
%***
\definecolor{UCDBlue}{RGB}{0, 108, 174}
\definecolor{UCDDarkBlue}{RGB}{16, 55, 96}

% ********************************************************************
% Using different fonts
% ********************************************************************
%\PassOptionsToPackage{doublespacing}{setspace}
\PassOptionsToPackage{onehalfspacing}{setspace}
\usepackage{setspace}
\setstretch{1.5}

\PassOptionsToPackage{switch}{lineno}
\usepackage{lineno}
\renewcommand\linenumberfont{\normalfont\fontsize{6}{0}\selectfont\color{Gray}}


\ifthenelse{\boolean{xetex}\OR\boolean{luatex}}
{
	% have to use sans serif fonts
	\RequirePackage{fontspec}
	\usepackage[varqu,varl]{zi4}% inconsolata typewriter
	\usepackage{amsmath,amsthm}
	\usepackage{libertinus}

	\newfontfamily\universityfont{Cabin}
	\newfontfamily\textuniversityfont{Cabin}
	\newfontfamily\texttitlepagefont{EB Garamond}
	\newfontfamily\textthesistitlefont{Latin Modern Sans}

	\setmainfont[Mapping=tex-text]{Libertinus Sans}
  \setmonofont{Fira Mono}
	\renewcommand{\familydefault}{\sfdefault}
}
{ % use Type 1 fonts with pdflatex
	\usepackage{librecaslon}
	\usepackage[T1]{fontenc}
	\usepackage{ebgaramond-maths}
	\usepackage{cabin}
	\usepackage{libertinus}
	\usepackage[varqu]{zi4}
	\usepackage{libertinust1math}

	\renewcommand*\familydefault{\sfdefault}  %% Only if the base font of the document is to be sans serif

	\DeclareTextFontCommand{\textuniversityfont}{
		\fontfamily{Cabin-TLF}\selectfont
	}

	\DeclareTextFontCommand{\texttitlepagefont}{
		\fontfamily{EBGaramond-TLF}\selectfont
	}
}


\usepackage[left=3.2cm, right=3.2cm, top=2cm, bottom=2.5cm, includehead, includefoot]{geometry}

\usepackage{algorithm}
\usepackage{algorithmicx}
\usepackage{algpseudocode}
% ****************************************************************************************************
% If you like the classicthesis, then I would appreciate a postcard.
% My address can be found in the file ClassicThesis.pdf. A collection
% of the postcards I received so far is available online at
% http://postcards.miede.de
% ****************************************************************************************************


% ****************************************************************************************************
% 0. Set the encoding of your files. UTF-8 is the only sensible encoding nowadays. If you can't read
% äöüßáéçèê∂åëæƒÏ€ then change the encoding setting in your editor, not the line below. If your editor
% does not support utf8 use another editor!
% ****************************************************************************************************
\PassOptionsToPackage{utf8}{inputenc}
\usepackage{inputenc}

\PassOptionsToPackage{T1}{fontenc} % T2A for cyrillics
\usepackage{fontenc}


% ****************************************************************************************************
% 1. Configure classicthesis for your needs here, e.g., remove "drafting" below
% in order to deactivate the time-stamp on the pages
% (see ClassicThesis.pdf for more information):
% ****************************************************************************************************
\PassOptionsToPackage{
	drafting=false,    % print version information on the bottom of the pages
	tocaligned=false, % the left column of the toc will be aligned (no indentation)
	dottedtoc=false,  % page numbers in ToC flushed right
	eulerchapternumbers=true, % use AMS Euler for chapter font (otherwise Palatino)
	linedheaders=false,       % chaper headers will have line above and beneath
	floatperchapter=true,     % numbering per chapter for all floats (i.e., Figure 1.1)
	eulermath=false,  % use awesome Euler fonts for mathematical formulae (only with pdfLaTeX)
	beramono=false,    % toggle a nice monospaced font (w/ bold)
	palatino=false,    % deactivate standard font for loading another one, see the last section at the end of this file for suggestions
	style=classicthesis % classicthesis, arsclassica
}{classicthesis}


% ****************************************************************************************************
% 2. Personal data and user ad-hoc commands (insert your own data here)
% ****************************************************************************************************
\newcommand{\myTitle}{Advances in Recommender Systems for Some Applications \xspace}
\newcommand{\mySubtitle}{A thesis about some recommender systems stuff\xspace}
\newcommand{\myDegree}{PhD\xspace}
\newcommand{\myStudentNumber}{18201111\xspace}
\newcommand{\myName}{Eolas MacDalta\xspace}
\newcommand{\myProf}{Put name here\xspace}
\newcommand{\myOtherProf}{Put name here\xspace}
\newcommand{\mySupervisor}{Prof. A.N Supervisor\xspace}
\newcommand{\myCoSupervisor}{Prof. A.N CoSupervisor\xspace}
\newcommand{\myFaculty}{Put data here\xspace}
\newcommand{\myHOS}{Prof. H. O'Scoil\xspace}
\newcommand{\myRSPA}{Prof. A. B. RSPHead\xspace}
\newcommand{\myRSPB}{Prof. B. C. RSPB\xspace}
\newcommand{\myDepartment}{School of Computer Science\xspace}
%\newcommand{\myTitleDeclaration}{A thesis submitted to University College Dublin in fulfilment of the requirements for the degree of Doctor of Philosophy\xspace}
\newcommand{\myTitleDeclaration}{This thesis is submitted to University College Dublin in fulfilment of the requirements for the degree of Doctor of Philosophy\xspace}
%\newcommand{\myFaculty}{Put data here\xspace}
\newcommand{\myUni}{University College Dublin\xspace}
\newcommand{\myLocation}{University College Dublin\xspace}
\newcommand{\myTime}{January, 2024\xspace}
\newcommand{\myVersion}{version 1.0\xspace}


% ********************************************************************
% Setup, finetuning, and useful commands
% ********************************************************************
\providecommand{\mLyX}{L\kern-.1667em\lower.25em\hbox{Y}\kern-.125emX\@}
\newcommand{\ie}{i.\,e.}
\newcommand{\Ie}{I.\,e.}
\newcommand{\eg}{e.\,g.}
\newcommand{\Eg}{E.\,g.}
\newcommand\cleartooddpage{\clearpage
	\ifodd\value{page}\else\null\clearpage\fi}
% ****************************************************************************************************


% ****************************************************************************************************
% 3. Loading some handy packages
% ****************************************************************************************************
% ********************************************************************
% Packages with options that might require adjustments
% ********************************************************************
\PassOptionsToPackage{english}{babel} % change this to your language(s), main language last
% Spanish languages need extra options in order to work with this template
%\PassOptionsToPackage{spanish,es-lcroman}{babel}
\usepackage{babel}

\usepackage{csquotes}
\PassOptionsToPackage{%
	%backend=biber,bibencoding=utf8, %instead of bibtex
	backend=bibtex8,bibencoding=ascii,%
	language=auto,%
	style=numeric-comp,%
	%style=authoryear-comp, % Author 1999, 2010
	%bibstyle=authoryear,dashed=false, % dashed: substitute rep. author with ---
	sorting=nyt, % name, year, title
	maxbibnames=10, % default: 3, et al.
	%backref=true,%
	natbib=true % natbib compatibility mode (\citep and \citet still work)
}{biblatex}
\usepackage{biblatex}

\PassOptionsToPackage{fleqn}{amsmath}       % math environments and more by the AMS
\usepackage{amsmath}

% ********************************************************************
% General useful packages
% ********************************************************************
\usepackage{graphicx} %
\usepackage[export]{adjustbox}
\usepackage{scrhack} % fix warnings when using KOMA with listings package
\usepackage{xspace} % to get the spacing after macros right
\PassOptionsToPackage{printonlyused,smaller}{acronym}
\usepackage{acronym} % nice macros for handling all acronyms in the thesis
%\renewcommand{\bflabel}[1]{{#1}\hfill} % fix the list of acronyms --> no longer working
%\renewcommand*{\acsfont}[1]{\textsc{#1}}
%\renewcommand*{\aclabelfont}[1]{\acsfont{#1}}
%\def\bflabel#1{{#1\hfill}}
\def\bflabel#1{{\acsfont{#1}\hfill}}
\def\aclabelfont#1{\acsfont{#1}}
% ****************************************************************************************************
%\usepackage{pgfplots} % External TikZ/PGF support (thanks to Andreas Nautsch)
%\usetikzlibrary{external}
%\tikzexternalize[mode=list and make, prefix=ext-tikz/]
% ****************************************************************************************************


% ****************************************************************************************************
% 4. Setup floats: tables, (sub)figures, and captions
% ****************************************************************************************************
\usepackage{tabularx} % better tables
\setlength{\extrarowheight}{3pt} % increase table row height
\newcommand{\tableheadline}[1]{\multicolumn{1}{l}{\spacedlowsmallcaps{#1}}}
\newcommand{\myfloatalign}{\centering} % to be used with each float for alignment
\usepackage{subfig}


\usepackage{blindtext}
% ****************************************************************************************************


% ****************************************************************************************************
% 5. Setup code listings
% ****************************************************************************************************
\usepackage{listings}
%\lstset{emph={trueIndex,root},emphstyle=\color{BlueViolet}}%\underbar} % for special keywords
\lstset{language=[LaTeX]Tex,%C++,
	morekeywords={PassOptionsToPackage,selectlanguage},
	keywordstyle=\color{RoyalBlue},%\bfseries,
	basicstyle=\small\ttfamily,
	%identifierstyle=\color{NavyBlue},
	commentstyle=\color{Green}\ttfamily,
	stringstyle=\rmfamily,
	numbers=none,%left,%
	numberstyle=\scriptsize,%\tiny
	stepnumber=5,
	numbersep=8pt,
	showstringspaces=false,
	breaklines=true,
	%frameround=ftff,
	%frame=single,
	belowcaptionskip=.75\baselineskip
	%frame=L
}

\definecolor{commentcolor}{rgb}{0.2,0.4,0.2}
\lstset{
    float,
    % By default, {listings} uses proportional-width fonts, but Source
    % Code Pro is nice enough to use on its own. Thus, we override basicstyle
    % as using the monospace font we declared earlier with {fontspec}.
    basicstyle=\ttfamily,
    % Also by default, everything is just black. Let's add some color to
    % comments.
    commentstyle=\small\color{commentcolor}\ttfamily,
    % The following few settings set the listings apart using a line
    % at the top and bottom, and subtle line numbers to the left.
    numbers=right,
    numberstyle=\tiny, stepnumber=5,
    numbersep=5pt,
    frame=lines,
    % Here's where the real magic is: mathescape. We want to catch both
    % Sphinx-style and LaTeX-style math escapes, which is a bit of a hack.
    % Just don't try and use this to typeset JavaScript code that touches
    % jQuery...
    mathescape=true,
    escapeinside={:math:`}{`},
    escapebegin=$,
    escapeend=$
}

% We'll go on and set a few options specific to Python, too, and encapsulate
% them into a {listings} style. This way, if we have a multilingual document,
% it will play nice.
\lstdefinestyle{oldpython}{
    language=Python,
    tabsize=4,
    showstringspaces=false
}

\definecolor{deepgreen}{rgb}{0,0.4,0}

\lstdefinestyle{python}{
		language=Python,
		basicstyle=\color{black}\ttfamily\footnotesize,
		stringstyle=\color{deepgreen}\slshape,
		commentstyle=\color{gray}\slshape,
		keywordstyle=\color{red}\textbf,
		emphstyle=\color{blue}\textbf,
		tabsize=2,
		%%%%%%%%%%%%%%%
		showstringspaces=false,
    emph={access,and,break,class,continue,def,del,elif,else,except,exec,finally,for,from,global,if,import,in,i s,lambda,not,or,pass,print,raise,return,try,while,as},
		upquote=true,
		morecomment=[s]{"""}{"""},
		literate=*
		{:}{{\textcolor{blue}:}}{1}%
		{=}{{\textcolor{blue}=}}{1}%
		{-}{{\textcolor{blue}-}}{1}%
		{+}{{\textcolor{blue}+}}{1}%
		{*}{{\textcolor{blue}*}}{1}%
		{!}{{\textcolor{blue}!}}{1}%
		{(}{{\textcolor{blue}(}}{1}%
		{)}{{\textcolor{blue})}}{1}%
		{[}{{\textcolor{blue}[}}{1}%
		{]}{{\textcolor{blue}]}}{1}%
		{<}{{\textcolor{blue}<}}{1}%
		{>}{{\textcolor{blue}>}}{1},%
		%%%%%%%%%%%%%%%%
		aboveskip=\baselineskip,
		xleftmargin=20pt, xrightmargin=15pt,
		frame=lines,
		numbers=left, numberstyle=\tiny
}

% ****************************************************************************************************




% ****************************************************************************************************
% 6. Last calls before the bar closes
% ****************************************************************************************************
\usepackage{afterpage}
% ********************************************************************
% Her Majesty herself
% ********************************************************************
\usepackage{classicthesis}

\clearscrheadfoot
\ohead[]{\headmark}
\rofoot[\pagemark]{\pagemark}
\lefoot[\pagemark]{\pagemark}

% ********************************************************************
% Fine-tune hyperreferences (hyperref should be called last)
% ********************************************************************
\hypersetup{%
	%draft, % hyperref's draft mode, for printing see below
	colorlinks=true, linktocpage=true, pdfstartpage=3, pdfstartview=FitV,%
	% uncomment the following line if you want to have black links (e.g., for printing)
	%colorlinks=false, linktocpage=false, pdfstartpage=3, pdfstartview=FitV, pdfborder={0 0 0},%
	breaklinks=true, pageanchor=true,%
	pdfpagemode=UseNone, %
	% pdfpagemode=UseOutlines,%
	plainpages=false, bookmarksnumbered, bookmarksopen=true, bookmarksopenlevel=1,%
	hypertexnames=true, pdfhighlight=/O,%nesting=true,%frenchlinks,%
	urlcolor=CTurl, linkcolor=CTlink, citecolor=CTcitation, %pagecolor=RoyalBlue,%
	%urlcolor=Black, linkcolor=Black, citecolor=Black, %pagecolor=Black,%
	pdftitle={\myTitle},%
	pdfauthor={\textcopyright\ \myName, \myUni, \myFaculty},%
	pdfsubject={},%
	pdfkeywords={},%
	pdfcreator={pdfLaTeX},%
	pdfproducer={LaTeX with hyperref and classicthesis}%
}


% ********************************************************************
% Setup autoreferences (hyperref and babel)
% ********************************************************************
% There are some issues regarding autorefnames
% http://www.tex.ac.uk/cgi-bin/texfaq2html?label=latexwords
% you have to redefine the macros for the
% language you use, e.g., american, ngerman
% (as chosen when loading babel/AtBeginDocument)
% ********************************************************************
\makeatletter
\@ifpackageloaded{babel}%
{%
	\addto\extrasenglish{%
		\renewcommand*{\figureautorefname}{Figure}%
		\renewcommand*{\tableautorefname}{Table}%
		\renewcommand*{\partautorefname}{Part}%
		\renewcommand*{\chapterautorefname}{Chapter}%
		\renewcommand*{\sectionautorefname}{Section}%
		\renewcommand*{\subsectionautorefname}{Section}%
		\renewcommand*{\subsubsectionautorefname}{Section}%
	}%
	\addto\extrasngerman{%
		\renewcommand*{\paragraphautorefname}{Absatz}%
		\renewcommand*{\subparagraphautorefname}{Unterabsatz}%
		\renewcommand*{\footnoteautorefname}{Fu\"snote}%
		\renewcommand*{\FancyVerbLineautorefname}{Zeile}%
		\renewcommand*{\theoremautorefname}{Theorem}%
		\renewcommand*{\appendixautorefname}{Anhang}%
		\renewcommand*{\equationautorefname}{Gleichung}%
		\renewcommand*{\itemautorefname}{Punkt}%
	}%
	% Fix to getting autorefs for subfigures right (thanks to Belinda Vogt for changing the definition)
	\providecommand{\subfigureautorefname}{\figureautorefname}%
}{\relax}
\makeatother


%
%
%
%\PassOptionsToPackage{}{glossaries}
\usepackage{glossaries}

% ********************************************************************
% Development Stuff
% ********************************************************************
\listfiles


% ****************************************************************************************************
% 7. Further adjustments (experimental)
% ****************************************************************************************************
% ********************************************************************
% Changing the text area
% ********************************************************************
%\areaset[current]{312pt}{761pt} % 686 (factor 2.2) + 33 head + 42 head \the\footskip
%\setlength{\marginparwidth}{7em}%
%\setlength{\marginparsep}{2em}%

%****
% colors
%***
\definecolor{UCDBlue}{RGB}{0, 108, 174}
\definecolor{UCDDarkBlue}{RGB}{16, 55, 96}

% ********************************************************************
% Using different fonts
% ********************************************************************
%\PassOptionsToPackage{doublespacing}{setspace}
\PassOptionsToPackage{onehalfspacing}{setspace}
\usepackage{setspace}
\setstretch{1.5}

\PassOptionsToPackage{switch}{lineno}
\usepackage{lineno}
\renewcommand\linenumberfont{\normalfont\fontsize{6}{0}\selectfont\color{Gray}}


\ifthenelse{\boolean{xetex}\OR\boolean{luatex}}
{
	% have to use sans serif fonts
	\RequirePackage{fontspec}
	\usepackage[varqu,varl]{zi4}% inconsolata typewriter
	\usepackage{amsmath,amsthm}
	\usepackage{libertinus}

	\newfontfamily\universityfont{Cabin}
	\newfontfamily\textuniversityfont{Cabin}
	\newfontfamily\texttitlepagefont{EB Garamond}
	\newfontfamily\textthesistitlefont{Latin Modern Sans}

	\setmainfont[Mapping=tex-text]{Libertinus Sans}
  \setmonofont{Fira Mono}
	\renewcommand{\familydefault}{\sfdefault}
}
{ % use Type 1 fonts with pdflatex
	\usepackage{librecaslon}
	\usepackage[T1]{fontenc}
	\usepackage{ebgaramond-maths}
	\usepackage{cabin}
	\usepackage{libertinus}
	\usepackage[varqu]{zi4}
	\usepackage{libertinust1math}

	\renewcommand*\familydefault{\sfdefault}  %% Only if the base font of the document is to be sans serif

	\DeclareTextFontCommand{\textuniversityfont}{
		\fontfamily{Cabin-TLF}\selectfont
	}

	\DeclareTextFontCommand{\texttitlepagefont}{
		\fontfamily{EBGaramond-TLF}\selectfont
	}
}


\usepackage[left=3.2cm, right=3.2cm, top=2cm, bottom=2.5cm, includehead, includefoot]{geometry}

\usepackage{algorithm}
\usepackage{algorithmicx}
\usepackage{algpseudocode}
% ****************************************************************************************************
% If you like the classicthesis, then I would appreciate a postcard.
% My address can be found in the file ClassicThesis.pdf. A collection
% of the postcards I received so far is available online at
% http://postcards.miede.de
% ****************************************************************************************************


% ****************************************************************************************************
% 0. Set the encoding of your files. UTF-8 is the only sensible encoding nowadays. If you can't read
% äöüßáéçèê∂åëæƒÏ€ then change the encoding setting in your editor, not the line below. If your editor
% does not support utf8 use another editor!
% ****************************************************************************************************
\PassOptionsToPackage{utf8}{inputenc}
\usepackage{inputenc}

\PassOptionsToPackage{T1}{fontenc} % T2A for cyrillics
\usepackage{fontenc}


% ****************************************************************************************************
% 1. Configure classicthesis for your needs here, e.g., remove "drafting" below
% in order to deactivate the time-stamp on the pages
% (see ClassicThesis.pdf for more information):
% ****************************************************************************************************
\PassOptionsToPackage{
	drafting=false,    % print version information on the bottom of the pages
	tocaligned=false, % the left column of the toc will be aligned (no indentation)
	dottedtoc=false,  % page numbers in ToC flushed right
	eulerchapternumbers=true, % use AMS Euler for chapter font (otherwise Palatino)
	linedheaders=false,       % chaper headers will have line above and beneath
	floatperchapter=true,     % numbering per chapter for all floats (i.e., Figure 1.1)
	eulermath=false,  % use awesome Euler fonts for mathematical formulae (only with pdfLaTeX)
	beramono=false,    % toggle a nice monospaced font (w/ bold)
	palatino=false,    % deactivate standard font for loading another one, see the last section at the end of this file for suggestions
	style=classicthesis % classicthesis, arsclassica
}{classicthesis}


% ****************************************************************************************************
% 2. Personal data and user ad-hoc commands (insert your own data here)
% ****************************************************************************************************
\newcommand{\myTitle}{Advances in Recommender Systems for Some Applications \xspace}
\newcommand{\mySubtitle}{A thesis about some recommender systems stuff\xspace}
\newcommand{\myDegree}{PhD\xspace}
\newcommand{\myStudentNumber}{18201111\xspace}
\newcommand{\myName}{Eolas MacDalta\xspace}
\newcommand{\myProf}{Put name here\xspace}
\newcommand{\myOtherProf}{Put name here\xspace}
\newcommand{\mySupervisor}{Prof. A.N Supervisor\xspace}
\newcommand{\myCoSupervisor}{Prof. A.N CoSupervisor\xspace}
\newcommand{\myFaculty}{Put data here\xspace}
\newcommand{\myHOS}{Prof. H. O'Scoil\xspace}
\newcommand{\myRSPA}{Prof. A. B. RSPHead\xspace}
\newcommand{\myRSPB}{Prof. B. C. RSPB\xspace}
\newcommand{\myDepartment}{School of Computer Science\xspace}
%\newcommand{\myTitleDeclaration}{A thesis submitted to University College Dublin in fulfilment of the requirements for the degree of Doctor of Philosophy\xspace}
\newcommand{\myTitleDeclaration}{This thesis is submitted to University College Dublin in fulfilment of the requirements for the degree of Doctor of Philosophy\xspace}
%\newcommand{\myFaculty}{Put data here\xspace}
\newcommand{\myUni}{University College Dublin\xspace}
\newcommand{\myLocation}{University College Dublin\xspace}
\newcommand{\myTime}{January, 2024\xspace}
\newcommand{\myVersion}{version 1.0\xspace}


% ********************************************************************
% Setup, finetuning, and useful commands
% ********************************************************************
\providecommand{\mLyX}{L\kern-.1667em\lower.25em\hbox{Y}\kern-.125emX\@}
\newcommand{\ie}{i.\,e.}
\newcommand{\Ie}{I.\,e.}
\newcommand{\eg}{e.\,g.}
\newcommand{\Eg}{E.\,g.}
\newcommand\cleartooddpage{\clearpage
	\ifodd\value{page}\else\null\clearpage\fi}
% ****************************************************************************************************


% ****************************************************************************************************
% 3. Loading some handy packages
% ****************************************************************************************************
% ********************************************************************
% Packages with options that might require adjustments
% ********************************************************************
\PassOptionsToPackage{english}{babel} % change this to your language(s), main language last
% Spanish languages need extra options in order to work with this template
%\PassOptionsToPackage{spanish,es-lcroman}{babel}
\usepackage{babel}

\usepackage{csquotes}
\PassOptionsToPackage{%
	%backend=biber,bibencoding=utf8, %instead of bibtex
	backend=bibtex8,bibencoding=ascii,%
	language=auto,%
	style=numeric-comp,%
	%style=authoryear-comp, % Author 1999, 2010
	%bibstyle=authoryear,dashed=false, % dashed: substitute rep. author with ---
	sorting=nyt, % name, year, title
	maxbibnames=10, % default: 3, et al.
	%backref=true,%
	natbib=true % natbib compatibility mode (\citep and \citet still work)
}{biblatex}
\usepackage{biblatex}

\PassOptionsToPackage{fleqn}{amsmath}       % math environments and more by the AMS
\usepackage{amsmath}

% ********************************************************************
% General useful packages
% ********************************************************************
\usepackage{graphicx} %
\usepackage[export]{adjustbox}
\usepackage{scrhack} % fix warnings when using KOMA with listings package
\usepackage{xspace} % to get the spacing after macros right
\PassOptionsToPackage{printonlyused,smaller}{acronym}
\usepackage{acronym} % nice macros for handling all acronyms in the thesis
%\renewcommand{\bflabel}[1]{{#1}\hfill} % fix the list of acronyms --> no longer working
%\renewcommand*{\acsfont}[1]{\textsc{#1}}
%\renewcommand*{\aclabelfont}[1]{\acsfont{#1}}
%\def\bflabel#1{{#1\hfill}}
\def\bflabel#1{{\acsfont{#1}\hfill}}
\def\aclabelfont#1{\acsfont{#1}}
% ****************************************************************************************************
%\usepackage{pgfplots} % External TikZ/PGF support (thanks to Andreas Nautsch)
%\usetikzlibrary{external}
%\tikzexternalize[mode=list and make, prefix=ext-tikz/]
% ****************************************************************************************************


% ****************************************************************************************************
% 4. Setup floats: tables, (sub)figures, and captions
% ****************************************************************************************************
\usepackage{tabularx} % better tables
\setlength{\extrarowheight}{3pt} % increase table row height
\newcommand{\tableheadline}[1]{\multicolumn{1}{l}{\spacedlowsmallcaps{#1}}}
\newcommand{\myfloatalign}{\centering} % to be used with each float for alignment
\usepackage{subfig}


\usepackage{blindtext}
% ****************************************************************************************************


% ****************************************************************************************************
% 5. Setup code listings
% ****************************************************************************************************
\usepackage{listings}
%\lstset{emph={trueIndex,root},emphstyle=\color{BlueViolet}}%\underbar} % for special keywords
\lstset{language=[LaTeX]Tex,%C++,
	morekeywords={PassOptionsToPackage,selectlanguage},
	keywordstyle=\color{RoyalBlue},%\bfseries,
	basicstyle=\small\ttfamily,
	%identifierstyle=\color{NavyBlue},
	commentstyle=\color{Green}\ttfamily,
	stringstyle=\rmfamily,
	numbers=none,%left,%
	numberstyle=\scriptsize,%\tiny
	stepnumber=5,
	numbersep=8pt,
	showstringspaces=false,
	breaklines=true,
	%frameround=ftff,
	%frame=single,
	belowcaptionskip=.75\baselineskip
	%frame=L
}

\definecolor{commentcolor}{rgb}{0.2,0.4,0.2}
\lstset{
    float,
    % By default, {listings} uses proportional-width fonts, but Source
    % Code Pro is nice enough to use on its own. Thus, we override basicstyle
    % as using the monospace font we declared earlier with {fontspec}.
    basicstyle=\ttfamily,
    % Also by default, everything is just black. Let's add some color to
    % comments.
    commentstyle=\small\color{commentcolor}\ttfamily,
    % The following few settings set the listings apart using a line
    % at the top and bottom, and subtle line numbers to the left.
    numbers=right,
    numberstyle=\tiny, stepnumber=5,
    numbersep=5pt,
    frame=lines,
    % Here's where the real magic is: mathescape. We want to catch both
    % Sphinx-style and LaTeX-style math escapes, which is a bit of a hack.
    % Just don't try and use this to typeset JavaScript code that touches
    % jQuery...
    mathescape=true,
    escapeinside={:math:`}{`},
    escapebegin=$,
    escapeend=$
}

% We'll go on and set a few options specific to Python, too, and encapsulate
% them into a {listings} style. This way, if we have a multilingual document,
% it will play nice.
\lstdefinestyle{oldpython}{
    language=Python,
    tabsize=4,
    showstringspaces=false
}

\definecolor{deepgreen}{rgb}{0,0.4,0}

\lstdefinestyle{python}{
		language=Python,
		basicstyle=\color{black}\ttfamily\footnotesize,
		stringstyle=\color{deepgreen}\slshape,
		commentstyle=\color{gray}\slshape,
		keywordstyle=\color{red}\textbf,
		emphstyle=\color{blue}\textbf,
		tabsize=2,
		%%%%%%%%%%%%%%%
		showstringspaces=false,
    emph={access,and,break,class,continue,def,del,elif,else,except,exec,finally,for,from,global,if,import,in,i s,lambda,not,or,pass,print,raise,return,try,while,as},
		upquote=true,
		morecomment=[s]{"""}{"""},
		literate=*
		{:}{{\textcolor{blue}:}}{1}%
		{=}{{\textcolor{blue}=}}{1}%
		{-}{{\textcolor{blue}-}}{1}%
		{+}{{\textcolor{blue}+}}{1}%
		{*}{{\textcolor{blue}*}}{1}%
		{!}{{\textcolor{blue}!}}{1}%
		{(}{{\textcolor{blue}(}}{1}%
		{)}{{\textcolor{blue})}}{1}%
		{[}{{\textcolor{blue}[}}{1}%
		{]}{{\textcolor{blue}]}}{1}%
		{<}{{\textcolor{blue}<}}{1}%
		{>}{{\textcolor{blue}>}}{1},%
		%%%%%%%%%%%%%%%%
		aboveskip=\baselineskip,
		xleftmargin=20pt, xrightmargin=15pt,
		frame=lines,
		numbers=left, numberstyle=\tiny
}

% ****************************************************************************************************




% ****************************************************************************************************
% 6. Last calls before the bar closes
% ****************************************************************************************************
\usepackage{afterpage}
% ********************************************************************
% Her Majesty herself
% ********************************************************************
\usepackage{classicthesis}

\clearscrheadfoot
\ohead[]{\headmark}
\rofoot[\pagemark]{\pagemark}
\lefoot[\pagemark]{\pagemark}

% ********************************************************************
% Fine-tune hyperreferences (hyperref should be called last)
% ********************************************************************
\hypersetup{%
	%draft, % hyperref's draft mode, for printing see below
	colorlinks=true, linktocpage=true, pdfstartpage=3, pdfstartview=FitV,%
	% uncomment the following line if you want to have black links (e.g., for printing)
	%colorlinks=false, linktocpage=false, pdfstartpage=3, pdfstartview=FitV, pdfborder={0 0 0},%
	breaklinks=true, pageanchor=true,%
	pdfpagemode=UseNone, %
	% pdfpagemode=UseOutlines,%
	plainpages=false, bookmarksnumbered, bookmarksopen=true, bookmarksopenlevel=1,%
	hypertexnames=true, pdfhighlight=/O,%nesting=true,%frenchlinks,%
	urlcolor=CTurl, linkcolor=CTlink, citecolor=CTcitation, %pagecolor=RoyalBlue,%
	%urlcolor=Black, linkcolor=Black, citecolor=Black, %pagecolor=Black,%
	pdftitle={\myTitle},%
	pdfauthor={\textcopyright\ \myName, \myUni, \myFaculty},%
	pdfsubject={},%
	pdfkeywords={},%
	pdfcreator={pdfLaTeX},%
	pdfproducer={LaTeX with hyperref and classicthesis}%
}


% ********************************************************************
% Setup autoreferences (hyperref and babel)
% ********************************************************************
% There are some issues regarding autorefnames
% http://www.tex.ac.uk/cgi-bin/texfaq2html?label=latexwords
% you have to redefine the macros for the
% language you use, e.g., american, ngerman
% (as chosen when loading babel/AtBeginDocument)
% ********************************************************************
\makeatletter
\@ifpackageloaded{babel}%
{%
	\addto\extrasenglish{%
		\renewcommand*{\figureautorefname}{Figure}%
		\renewcommand*{\tableautorefname}{Table}%
		\renewcommand*{\partautorefname}{Part}%
		\renewcommand*{\chapterautorefname}{Chapter}%
		\renewcommand*{\sectionautorefname}{Section}%
		\renewcommand*{\subsectionautorefname}{Section}%
		\renewcommand*{\subsubsectionautorefname}{Section}%
	}%
	\addto\extrasngerman{%
		\renewcommand*{\paragraphautorefname}{Absatz}%
		\renewcommand*{\subparagraphautorefname}{Unterabsatz}%
		\renewcommand*{\footnoteautorefname}{Fu\"snote}%
		\renewcommand*{\FancyVerbLineautorefname}{Zeile}%
		\renewcommand*{\theoremautorefname}{Theorem}%
		\renewcommand*{\appendixautorefname}{Anhang}%
		\renewcommand*{\equationautorefname}{Gleichung}%
		\renewcommand*{\itemautorefname}{Punkt}%
	}%
	% Fix to getting autorefs for subfigures right (thanks to Belinda Vogt for changing the definition)
	\providecommand{\subfigureautorefname}{\figureautorefname}%
}{\relax}
\makeatother


%
%
%
%\PassOptionsToPackage{}{glossaries}
\usepackage{glossaries}

% ********************************************************************
% Development Stuff
% ********************************************************************
\listfiles


% ****************************************************************************************************
% 7. Further adjustments (experimental)
% ****************************************************************************************************
% ********************************************************************
% Changing the text area
% ********************************************************************
%\areaset[current]{312pt}{761pt} % 686 (factor 2.2) + 33 head + 42 head \the\footskip
%\setlength{\marginparwidth}{7em}%
%\setlength{\marginparsep}{2em}%

%****
% colors
%***
\definecolor{UCDBlue}{RGB}{0, 108, 174}
\definecolor{UCDDarkBlue}{RGB}{16, 55, 96}

% ********************************************************************
% Using different fonts
% ********************************************************************
%\PassOptionsToPackage{doublespacing}{setspace}
\PassOptionsToPackage{onehalfspacing}{setspace}
\usepackage{setspace}
\setstretch{1.5}

\PassOptionsToPackage{switch}{lineno}
\usepackage{lineno}
\renewcommand\linenumberfont{\normalfont\fontsize{6}{0}\selectfont\color{Gray}}


\ifthenelse{\boolean{xetex}\OR\boolean{luatex}}
{
	% have to use sans serif fonts
	\RequirePackage{fontspec}
	\usepackage[varqu,varl]{zi4}% inconsolata typewriter
	\usepackage{amsmath,amsthm}
	\usepackage{libertinus}

	\newfontfamily\universityfont{Cabin}
	\newfontfamily\textuniversityfont{Cabin}
	\newfontfamily\texttitlepagefont{EB Garamond}
	\newfontfamily\textthesistitlefont{Latin Modern Sans}

	\setmainfont[Mapping=tex-text]{Libertinus Sans}
  \setmonofont{Fira Mono}
	\renewcommand{\familydefault}{\sfdefault}
}
{ % use Type 1 fonts with pdflatex
	\usepackage{librecaslon}
	\usepackage[T1]{fontenc}
	\usepackage{ebgaramond-maths}
	\usepackage{cabin}
	\usepackage{libertinus}
	\usepackage[varqu]{zi4}
	\usepackage{libertinust1math}

	\renewcommand*\familydefault{\sfdefault}  %% Only if the base font of the document is to be sans serif

	\DeclareTextFontCommand{\textuniversityfont}{
		\fontfamily{Cabin-TLF}\selectfont
	}

	\DeclareTextFontCommand{\texttitlepagefont}{
		\fontfamily{EBGaramond-TLF}\selectfont
	}
}


\usepackage[left=3.2cm, right=3.2cm, top=2cm, bottom=2.5cm, includehead, includefoot]{geometry}

\usepackage{algorithm}
\usepackage{algorithmicx}
\usepackage{algpseudocode}